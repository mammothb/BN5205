\documentclass[11pt, a4paper]{article}

\usepackage{amsmath}
\usepackage{bm}
\usepackage[margin=1in]{geometry}
\usepackage{graphicx}
\usepackage[parfill]{parskip}
\usepackage{pgfplots}

\newcommand{\Psim}{\ensuremath{\Psi_m(\xi)}}
\newcommand{\Psin}{\ensuremath{\Psi_n(\xi)}}

\begin{document}
  \paragraph{name,matric} BN5205 Finite Element Assignment
  \section{Getting the weak formulation}
  Governing equation:
  \begin{equation}
  \frac{\partial u}{\partial t} = \nabla \cdot (D \nabla u) + u
  \end{equation}
  Boundary condition:
  \begin{equation}
  D \frac{\partial u}{\partial n} |_{\Gamma} = g
  \end{equation}
  where $g = 0$.
  
  Galerkin weighted residual method:
  \begin{equation*}
  \int_{\Omega} \left(\frac{\partial u}{\partial t} - \nabla \cdot (D \nabla u) 
  - u \right) \omega \: d\Omega = 0
  \end{equation*}
  Since $D = 0.1$ is a scalar value:
  \begin{equation} \label{eq:mwr}
  \int_{\Omega} \left(\frac{\partial u}{\partial t} - D \nabla^2 u - u 
  \right) \omega \: d\Omega = 0
  \end{equation}
  Green-Gauss theorem:
  \begin{equation} \label{eq:green-gauss}
  \int_{\Omega} (f \nabla \cdot (h\nabla g) + \nabla f \cdot (h \nabla g)) \: 
  d\Omega = \int_{\Gamma} f h \frac{\partial g}{\partial n} \: d\Gamma
  \end{equation}
  Substitute $f = \omega$, $g = u$, and $h = D$ into Eq. \eqref{eq:green-gauss}
  \begin{equation} \label{eq:green-gauss-subbed}
  \int_{\Omega} D\nabla^2 u \cdot \omega \: d\Omega = -\int_{\Omega} D\nabla 
  u \cdot \nabla \omega \: d\Omega + \int_{\Gamma} D \frac{\partial 
  u}{\partial n} \omega \: d\Gamma
  \end{equation}
  Splitting the integral and using Eq. \eqref{eq:green-gauss-subbed} in Eq. 
  \eqref{eq:mwr} to obtain weak formulation:
  \begin{equation} \label{eq:weak}
  \int_{\Omega} \frac{\partial u}{\partial t} \: \omega \: d\Omega + 
  \int_{\Omega} D\nabla u \cdot \nabla \omega \: d\Omega - \int_{\Omega} u \: 
  \omega \: d\Omega = \int_{\Gamma} D \frac{\partial u}{\partial n} \omega \: 
  d\Gamma
  \end{equation}
  Let $u = \Psin u_n$ and $\omega = \Psim$. Represent Eq. 
  \eqref{eq:weak} as a system of first order ODEs:
  \begin{equation}
  \mathbf{M} \frac{\partial u}{\partial t} + \mathbf{K}u = \mathbf{f}
  \end{equation}
  where $\mathbf{M}$ is the global mass matrix, its element contributions are 
  given by
  \begin{equation}
  M_{mn}^e = \int_{0}^{1}\Psin \Psim J \: d\xi
  \end{equation}
  $\mathbf{K}$ is the global stiffness matrix, its element contributions are 
  given by
  \begin{equation}
  K_{mn}^e = \int_{0}^{1} \left(D \frac{\partial \Psin}{\partial \xi_i} 
  \frac{\partial \xi_i}{\partial x_k} \frac{\partial \Psim}{\partial \xi_j} 
  \frac{\partial \xi_j}{\partial x_k} -\Psin \Psim \right)\: J \: d\bm{\xi}
  \end{equation}
  and $\mathbf{f}$ is the RHS vector, its element contributions are given by
  \begin{equation}
  f_m = \int_{\Gamma} g \Psim \: d\Gamma
  \end{equation}
  where $g=0$.
  \section{Choice of time and space step}
  The initial bacterial concentration creates a sharp spike in the middle of 
  the domain. Therefore, fine mesh is needed to accurately model approximate 
  the sharp change in density surrounding the spike. A simple mesh convergence 
  analysis was conducted (Fig. \ref{fig:space_step}). The finest mesh tested 
  was $80 \times 80$ due to memory limitation.
  \begin{figure}[h]
    \centering
    \begin{tikzpicture} %pgfplot
    \begin{axis}[
      scale=1.0,
      xmin=0,
      ymin=0,
      xlabel={Number of element per side},
      ylabel={Bacterial concentration $u(4,4)$}
    ]
    \addplot[blue, mark=square] table [x=elem, y=u_mid1]{mesh_conv.dat};
    \end{axis}
    \end{tikzpicture}
    \caption{Mesh convergence analysis \label{fig:space_step}}
  \end{figure}
  
  As the difference between the results obtained $80 \times 80$ and $64 \times 
  64$ mesh is around $50\%$, a smaller space step is necessary to accurately 
  model bacteria growth.
  
  As the Crank-Nicolson-Galerkin scheme was used, the time step size has no 
  effect on the stability of the simulation. However, the time step size was 
  set to be $\Delta t = 0.05$ in a $64 \times 64$ mesh so that the Courant 
  condition 
  \begin{equation}
  D \frac{\Delta t}{\Delta x^2} \leq \frac{1}{2}
  \end{equation}
  could be satisfied to ensure numerical accuracy.
\end{document}